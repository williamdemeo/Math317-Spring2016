\documentclass[11pt]{amsart}
%\documentclass[11pt]{article}
%%%%%%%%%%%%%%%%%%%%%%%%%%%%%%%%%%%%%%%%%%%%%%%
\usepackage{tikz-cd}
\usepackage{scalefnt}
\usepackage{amsmath,amsthm,amssymb}
\usepackage[mathcal]{euscript}
\usepackage{mathtools}
\usepackage{etoolbox}
\usepackage{fancyhdr}
 \usepackage{xcolor}
\usepackage[colorlinks=true,urlcolor=blue,linkcolor=blue,citecolor=blue]{hyperref}
%\usepackage[colorlinks=true,urlcolor=black,linkcolor=black,citecolor=black]{hyperref}
\usepackage{xspace}
\usepackage{comment}
\usepackage{url} % for url in bib entries
\usepackage{mathrsfs}
\usepackage{enumerate}


%% \usepackage{amscd,amssymb}  % amsthm, amsmath are included by default
%% \usepackage{latexsym,stmaryrd,mathrsfs,enumerate,scalefnt,ifthen}
%% \usepackage[mathcal]{euscript}
%% \usepackage{amsmath,amsthm,amssymb}
%% \usepackage{mathtools}
%% \usepackage{etoolbox}

%% %% for long mapsto arrows
%% %% centernot,
%% \usepackage{mathtools,stmaryrd}
%% \makeatletter

\newcommand{\xMapsto}[2][]{\ext@arrow 0599{\Mapstofill@}{#1}{#2}}
\def\Mapstofill@{\arrowfill@{\Mapstochar\Relbar}\Relbar\Rightarrow}
\makeatother
\newcommand{\xlmapsto}[2][]{\xmapsto[\phantom{x}#1\phantom{x}]{\phantom{x}#2\phantom{x}}}
\newcommand{\xlMapsto}[2][]{\xMapsto[\phantom{x}#1\phantom{x}]{\phantom{x}#2\phantom{x}}}
%% use it like this:
%%  \xmapsto[below]{above}  (``below'' goes below the arrow; ``above'' goes above)
%%  \xlmapsto[below]{above}
%%  \xMapsto[A]{sdfkjhsdf}  (double arrow version)
%%  \xlMapsto[A]{sdfkjhsdf}  (double arrow version)



\usepackage[style = ieee, urldate = comp]{biblatex}


%%==============================================================================
%% ---------- DEFINITIONS  -----------
% New terms will be typeset as follows:
% \newcommand{\defn}[1]{{\bf \emph{#1}}}    % bold and italic
% (later we can change this to just bold, or just italic if we want)
% \newcommand{\defn}[1]{{\bf #1}}  % bold
% \newcommand{\defn}[1]{\emph{#1}}  % italic

% Both definitions and indexing (from the commutator book):
%%%--------------------------------------------------------
%%% (modified from the free lattice book)
%%% \defn usage:
%%%     This writes its argument in bold italics and also calls \index
%%%     for the index. eg.
%%%
%%%     \defn{join cover}
%%%
%%%     It has an optional second argument to specify the index
%%%     entry more precisely which works just like \index.
%%%
%%%     If you are writing about join covers (but they were defined
%%%     earlier, you can use \index{join cover}. This puts
%%%     nothing in the text, but makes an index entry.
%%%
\newcommand{\indexit}[1]{\index{#1|textit}}
\def\defn#1{\gdef\defnstring{#1}%
  %%% original:   \xdef\dodefnii{{\noexpand\em
  %% \xdef\dodefnii{{\noexpand\bfseries\noexpand\em
  \xdef\dodefnii{{\noexpand\em
       \defnstring}\noexpand\indexit{\defnstring}\noexpand\makeatother}%
  \futurelet\nextthing\dodefn}
\def\dodefn{%
  \ifx\nextthing[\let\next=\dodefni
    \else\let\next=\dodefnii\fi
  \makeatletter
  \next}

\def\dodefni[#1]{%
  {\bfseries\em\defnstring}%
  \indexit{#1}%
  \makeatother}

%\newcommand{\defn}[1]{{\bf #1}}  %% \defn is now defined above so 
%%                  that each use is accompanied by an index entry.
%%                  To define something without an index entry, use
\newcommand{\defb}[1]{{\bf #1}}  %%       \defb (for bold)
\newcommand{\defi}[1]{{\it #1}}  %%       \defi (for italicised)
\newcommand{\defbi}[1]{{\bfseries \em  #1}}  %%       \defbi (for bold and ital)


%% CATEGORIES
\newcommand{\cat}[1]{\ensuremath{\mathcal{#1}}\xspace}
\newcommand{\R}{\ensuremath{\mathbb R}}
\newcommand{\C}{\ensuremath{\mathbb C}}
\newcommand{\Z}{\ensuremath{\mathbb Z}}
\newcommand{\sL}{\ensuremath{\mathcal L}}
\newcommand{\sT}{\ensuremath{\mathcal T}}
\newcommand{\sE}{\ensuremath{\mathcal E}}
\newcommand{\nn}{\ensuremath{\underline{n}}}
\newcommand{\cC}{\cat{C}}
\newcommand{\cD}{\cat{D}}
\newcommand{\cE}{\cat{E}}
\newcommand{\cF}{\cat{F}}
\newcommand{\cM}{\cat{M}}
\newcommand{\src}{\ensuremath{\operatorname{src}}}
\newcommand{\tar}{\ensuremath{\operatorname{tar}}}

\newcommand\vE{\ensuremath{\operatorname{E}}}
\newcommand\vfour{\ensuremath{\underline{4}}}

\newcommand\Set{\ensuremath{\mathbf{Set}}}        %% Sets and total functions
\newcommand\Par{\ensuremath{\mathbf{Par}}}        %% Sets and partial functions
\newcommand\Mon{\ensuremath{\mathbf{Mon}}}        %% Monoids
\newcommand\ocpo{\ensuremath{\omega\mathbf{cpo}}} %% omega-chain cocomplete posets
\newcommand\dcpo{\ensuremath{\mathbf{dcpo}}}      %% directed-cocomplete posets
\newcommand\Pre{\ensuremath{\mathbf{Pre}}}        %% Preorders
\newcommand\Poset{\ensuremath{\mathbf{Poset}}}    %% Posets
\newcommand\Grp{\ensuremath{\mathbf{Grp}}}        %% Lattices
\newcommand\Lat{\ensuremath{\mathbf{Lat}}}        %% Lattices
\newcommand\CLat{\ensuremath{\mathbf{CLat}}}      %% Compelete lattices
\newcommand\ACLat{\ensuremath{\mathbf{ACLat}}}    %% Algebraic compelete lattices
\newcommand\BLat{\ensuremath{\mathbf{BLat}}}      %% Boolean lattices
\newcommand\HLat{\ensuremath{\mathbf{HLat}}}      %% Heyting lattices
\newcommand\SDom{\ensuremath{\mathbf{SDom}}}      %% Scott domains

%% Alternatively, we could use the more traditional double underline.
%% \def\dunderline#1{\underline{\underline{#1}}}
%% \newcommand\Set{\ensuremath{\dunderline{\mathrm{Set}}}}

%%==============================================================================

%%%%%%%%%%%%%%%%%%%%%%%%%%%%%%%%%%%%%%%%
% Acronyms
%%%%%%%%%%%%%%%%%%%%%%%%%%%%%%%%%%%%%%%%
%% \usepackage[acronym, shortcuts]{glossaries}
%\usepackage[smaller]{acro}
\usepackage[smaller]{acronym}

%% \acs{CSP} -- short version of the acronym\\
%% \acl{CSP} -- expanded acronym without mentioning the acronym.\\
%% \acp{CSP} -- plurals.\\
%% \acfp{CSP} -- long forms into plurals.\\
%% \acsp{CSP} -- short form into a plural.\\
%% \aclp{CSP} -- long form into a plural.\\
%% \acfi{CSP} -- Full Name acronym in italics and abbreviated form in upshape.\\
%% \acsu{CSP} -- short form of the acronym and marks it as used.\\
%% \aclu{CSP} -- Prints the long form of the acronym and marks it as used.\\

\usepackage{xspace}

\acrodef{sat}[SAT]{satisfiability}
\acrodef{dsp}[DSP]{discrete signal processing}
\acrodef{nae}[NAE]{not-all-equal}
\acrodef{ctb}[CTB]{cube term blocker}
\acrodef{tct}[TCT]{tame congruence theory}
\acrodef{wnu}[WNU]{weak near-unanimity}
\acrodef{CSP}[CSP]{constraint satisfaction problem}
\acrodef{cib}[CIB]{commutative idempotent binar}
\acrodef{NP}[NP]{nondeterministic polynomial time}
\acrodef{P}[P]{polynomial time}
\acrodef{PeqNP}[P $ = $ NP]{P is NP}
\acrodef{PneqNP}[P $ \neq $ NP]{P is not NP}
\newcommand{\bool}{\ensuremath{\mathtt{Bool}}}
\newcommand{\true}{\ensuremath{\mathtt{true}}}
\newcommand{\false}{\ensuremath{\mathtt{false}}}
\newcommand{\CSP}{\ensuremath{\operatorname{CSP}}}
\newcommand{\slt}{\ensuremath{\mathbf S_2}}
\newcommand{\wnu}{\acs{wnu}\xspace}
\newcommand{\tct}{\acs{tct}\xspace}
\newcommand{\ctb}{\acs{ctb}\xspace}
\newcommand{\csp}{\acs{CSP}\xspace}
\newcommand{\dsp}{\acs{dsp}\xspace}
\newcommand{\sat}{\acs{sat}\xspace}
\newcommand{\nae}{\acs{nae}\xspace}
\newcommand{\csps}{\acsp{CSP}\xspace}
\newcommand{\cib}{\acs{cib}\xspace}
\newcommand{\cibs}{\acsp{cib}\xspace}
\newcommand{\PeqNP}{\acs{PeqNP}\xspace}
\newcommand{\PneqNP}{\acs{PneqNP}\xspace}
\newcommand{\NPcomplete}{\acs{NP}-complete\xspace}
\newcommand{\NP}{\acs{NP}\xspace}
\renewcommand{\P}{\acs{P}\xspace}
%\newcommand{\Maroti}{\textsf{Mar\'{o}ti}\xspace}
\usepackage{scalefnt}
\usepackage{tikz}
\usepackage{color}
\usepackage[margin=1in]{geometry}
\usetikzlibrary{calc}
\newcommand{\mfA}{\ensuremath{\mathfrak{A}}}
\newcommand{\mfX}{\ensuremath{\mathfrak{X}}}
\newcommand{\lb}{\ensuremath{\llbracket}}
\newcommand{\rb}{\ensuremath{\rrbracket}}
\newcommand{\id}{\ensuremath{\operatorname{id}}}
\newcommand{\fin}{\ensuremath{\operatorname{fin}}}
\newcommand{\Eq}{\ensuremath{\operatorname{Eq}}}
\newcommand{\Rel}{\ensuremath{\operatorname{Rel}}}
\newcommand{\Proj}{\ensuremath{\operatorname{Proj}}}
\newcommand{\arity}{\ensuremath{\operatorname{arity}}}
\newcommand{\Op}{\ensuremath{\operatorname{Op}}}
\newcommand{\alg}[1]{\ensuremath{\mathbf{#1}}}
\newcommand{\Sub}{\ensuremath{\operatorname{Sub}}}
\newcommand{\Con}{\ensuremath{\operatorname{Con}}}
\newcommand{\Clo}{\ensuremath{\operatorname{Clo}}}
\newcommand{\Pol}{\ensuremath{\operatorname{Pol}}}
\newcommand{\Poly}{\ensuremath{\operatorname{Poly}}}
\newcommand{\V}{\ensuremath{\operatorname{V}}}

\usepackage{bm}
\newcommand{\N}{\ensuremath{\mathbb{N}}}
\newcommand{\SDwedge}{\ensuremath{\mbox{SD}_\wedge}}

\newcommand{\<}{\ensuremath{\langle}}
\renewcommand{\>}{\ensuremath{\rangle}}
\newcommand{\bR}{\ensuremath{\mathbf{R}}}
\newcommand{\sansM}{\ensuremath{\mathsf{M}}}
\newcommand{\sansA}{\ensuremath{\mathsf{A}}}
\newcommand{\sansC}{\ensuremath{\mathsf{C}}}
\newcommand{\sansH}{\ensuremath{\mathsf{H}}}
\newcommand{\sansO}{\ensuremath{\mathsf{O}}} % ALL FINITARY OPERATIONS
\newcommand{\sansP}{\ensuremath{\mathsf{P}}}
\newcommand{\sansPoly}{\ensuremath{\mathsf{poly}}}
\newcommand{\sansRel}{\ensuremath{\mathsf{rel}}}
\newcommand{\sansClo}{\ensuremath{\mathsf{clo}}}
\newcommand{\sansR}{\ensuremath{\mathsf{R}}} % ALL FINITARY RELATIONS
\newcommand{\sansS}{\ensuremath{\mathsf{S}}}
\newcommand{\sansT}{\ensuremath{\mathsf{T}}}
\newcommand{\bA}{\ensuremath{\mathbf{A}}}
\newcommand{\bD}{\ensuremath{\mathbf{D}}}
\newcommand{\bF}{\ensuremath{\mathbf{F}}}
\newcommand{\bM}{\ensuremath{\mathbf{M}}}
\newcommand{\bP}{\ensuremath{\mathbf{P}}}
\newcommand{\br}{\ensuremath{\mathbf{r}}}
\newcommand{\bbA}{\ensuremath{\mathbb{A}}}
\newcommand{\bbB}{\ensuremath{\mathbb{B}}}
\newcommand{\bbC}{\ensuremath{\mathbb{C}}}
\newcommand{\bbD}{\ensuremath{\mathbb{D}}}
\newcommand{\bbS}{\ensuremath{\mathbb{S}}}
\newcommand{\bbV}{\ensuremath{\mathbb{V}}}
\newcommand{\bs}{\ensuremath{\mathbf{s}}}
\newcommand{\sA}{\ensuremath{\mathscr{A}}}
\newcommand{\sB}{\ensuremath{\mathcal{B}}}
\newcommand{\sS}{\ensuremath{\mathscr{S}}}
\newcommand{\bB}{\ensuremath{\mathbf{B}}}
\newcommand{\bC}{\ensuremath{\mathbf{C}}}
%\newcommand{\cC}{\ensuremath{\mathcal{C}}}
\newcommand{\cS}{\ensuremath{\mathcal{S}}}
\newcommand{\cP}{\ensuremath{\mathcal{P}}}
\newcommand{\cI}{\ensuremath{\mathcal{I}}}
\newcommand{\sI}{\ensuremath{\mathscr{I}}}
\newcommand{\sM}{\ensuremath{\mathscr{M}}}
\newcommand{\sR}{\ensuremath{\mathscr{R}}}
\newcommand{\bS}{\ensuremath{\mathbf{S}}}
\newcommand{\bx}{\ensuremath{\mathbf{x}}}
\newcommand{\power}[1]{\ensuremath{\mathscr{P}(#1)}}
\newcommand{\balpha}{\ensuremath{\boldsymbol{\alpha}}}
\newcommand{\bbeta}{\ensuremath{\boldsymbol{\beta}}}
\newcommand{\bgamma}{\ensuremath{\boldsymbol{\gamma}}}
\newcommand{\bdelta}{\ensuremath{\boldsymbol{\delta}}}
\newcommand{\bepsilon}{\ensuremath{\boldsymbol{\epsilon}}}
\newcommand{\AAn}{\ensuremath{A^{(A^n)}}}
\newcommand{\AAm}{\ensuremath{A^{(A^m)}}}
\newcommand{\AAk}{\ensuremath{A^{(A^k)}}}
\newcommand{\meet}{\ensuremath{\wedge}}
\newcommand{\Meet}{\ensuremath{\bigwedge}}
\renewcommand{\Join}{\ensuremath{\bigvee}}
\newcommand{\join}{\ensuremath{\vee}}
\newcommand{\relR}{\ensuremath{\mathrel{R}}}
\renewcommand{\vec}[1]{\mathbf{#1}}
         \newcommand\va{\vec{a}}
         \newcommand\vf{\vec{f}}
         \newcommand\vu{\vec{u}}
         \newcommand\vv{\vec{v}}
         \newcommand\vw{\vec{w}}
         \newcommand\vx{\vec{x}}
         \newcommand\vy{\vec{y}}
         \newcommand\vz{\vec{z}}
         \newcommand\vzero{\vec{0}}
         \newcommand\sC{\ensuremath{\mathcal C}}
         \newcommand\sV{\ensuremath{\mathcal V}}
         \newcommand\sW{\ensuremath{\mathcal W}}
         \newcommand\sP{\ensuremath{\mathcal P}}
         \newcommand\Span{\ensuremath{\operatorname{Span}}}

%%////////////////////////////////////////////////////////////////////////////////
%% Theorem styles
\numberwithin{equation}{section}
\theoremstyle{plain}
\newtheorem{theorem}{Theorem}[section]
\newtheorem{lemma}[theorem]{Lemma}
\newtheorem{proposition}[theorem]{Proposition}
\newtheorem{prop}[theorem]{Proposition}
\newtheorem{corollary}[theorem]{Corollary}
\newtheorem{claim}[theorem]{Claim}
\theoremstyle{definition}
\newtheorem{definition}[theorem]{Definition}
\newtheorem{notation}[theorem]{Notation}
\newtheorem{remark}[theorem]{Remark}
\newtheorem{example}[theorem]{Example}
\newtheorem{examples}[theorem]{Examples}
\newtheorem{exercise}{Exercise}
\newtheorem*{remarks}{Remarks}
\newtheorem{fact}{Fact}
\newtheorem*{obs}{Observation}
%\newtheorem{definition}{Definition}
%\newtheorem{example}{Example}
%\newtheorem{theorem}{Theorem}
%\newtheorem{lemma}{Lemma}


\usepackage{setspace}
\onehalfspacing

\addbibresource{refs.bib}

%% All exercises can be excluded by setting the exercises variable to false.
\newboolean{exercises}
\setboolean{exercises}{true}
%% To remove exercises, uncomment `\setboolean{exercises}{false}`
%%\setboolean{exercises}{false}  % set to false to exclude exercises
\newcommand{\Exercise}[1]{\ifthenelse{\boolean{exercises}}{\begin{exercise}#1\end{exercise}}{}}


\makeindex 


%%% This enables markdown style bold and italic. %%%
\makeatletter
\def\starparse{\@ifnextchar*{\bfstarx}{\itstar}}
\def\bfstarx#1{\@ifnextchar*{\bfitstar\@gobble}{\bfstar}}
\makeatother
\def\itstar#1*{\textit{#1}\starON}
\def\bfstar#1**{\textbf{#1}\starON}
\def\bfitstar#1***{\textbf{\textit{#1}}\starON}
\def\starON{\catcode`\*=\active}
\def\starOFF{\catcode`\*=12}
\starON
\def*{\starOFF \starparse}
\starOFF

\begin{document}
\date{\today \\
  The latest draft is available at
  \href{https://github.com/williamdemeo}{github.com/williamdemeo}
}
\title{Math 317 Note: Change Of Basis}
\author{William DeMeo}

\maketitle
\starON

\begin{abstract}
This handout explains how to convert from one basis representation of a
linear transformation to another.  It assumes a lot of the reader.  In
particular, the reader is expected to already know about change-of-basis
formulae, and this handout is intended primarily for reference 
when reviewing for the final exam.
\end{abstract}

\section{Changing basis representations of vectors}
Let $V$ be an $n$-dimensional vector space and suppose 
$\sV = \{\vv_1, \vv_2, \dots, \vv_n\}$ and 
$\sV' = \{\vv_1', \vv_2', \dots, \vv_n'\}$ are two bases 
for $V$.  
Recall that the $\sV$-basis representation of a vector $\vx\in V$ is
defined to be the $n$-tuple $[\vx]_\sV = (c_1, c_2, \dots, c_n)$
of coefficients satisfying
\[
\vx = c_1 \vv_1 + c_2 \vv_2 + \cdots +c_n \vv_n.
\]
Similarly, the representation of $\vx$ in the basis $\sV'$ is the $n$-tuple
$[\vx]_{\sV'} = (d_1, d_2, \dots, d_n)$
of coefficients satisfying
\[
\vx = d_1 \vv_1' + d_2 \vv_2' + \cdots +d_n \vv_n'.
\]
The first question we ask is, if we are given $[\vx]_\sV$, how to compute 
$[\vx]_{\sV'}$?  And if we are given $[\vx]_{\sV'}$, how to compute 
$[\vx]_{\sV}$?  In this first section, we review how to do this.  In the next
section we will take up the similar question for representations of linear transformations.

To change between different basis representations of vectors in $\sV$, 
we construct the \defn{change-of-basis matrix} $P$ from $\sV'$ to $\sV$
as follows: write each $\vv_i'\in \sV'$ as a linear combination of the
vectors in the first basis,
\begin{equation}
  \label{eq:1}
\vv_i' = \alpha_{1i}\vv_1 + \alpha_{2i}\vv_2 + \cdots + \alpha_{ni}\vv_n.
\end{equation}
That is, we compute the coefficients, $[\vv_i']_\sV = (\alpha_{1i}, \alpha_{2i}, \dots, \alpha_{ni})$.
%% \begin{align}
%% \vv_1' &= \alpha_{11}\vv_1 + \alpha_{21}\vv_2 + \cdots + \alpha_{n1}\vv_n\nonumber\\
%% \vv_2' &= \alpha_{12}\vv_1 + \alpha_{22}\vv_2 + \cdots + \alpha_{n2}\vv_n\nonumber\\
%%  &\vdots\nonumber\\
%% \vv_n' &= \alpha_{1n}\vv_1 + \alpha_{2n}\vv_2 + \cdots + \alpha_{nn}\vv_n.\nonumber
%% \end{align}
%% Another way to say this is that we want to compute the $\sV$-basis 
%% representation of each vector $\vv_i'$ in the new basis:
%% \begin{align}
%% [\vv_2']_\sV &= (\alpha_{12}, \alpha_{22}, \dots, \alpha_{n2})\nonumber\\
%%  & \quad \vdots\nonumber\\
%% [\vv_n']_\sV &= (\alpha_{1n}, \alpha_{2n}, \dots, \alpha_{nn}).\nonumber
%% \end{align}
%% At this point, 
It is important to notice that equation~(\ref{eq:1})
%% is given in terms of the $\sV$-basis representation of the 
%% vector $\vv_i'$ is given by
is equivalent to the following matrix vector multiplication:
\[
\vv_i' = 
\begin{bmatrix}
| & | & & |\\
\vv_1 & \vv_2 & \cdots & \vv_n\\
| & | & & |
\end{bmatrix}
\begin{bmatrix}
\alpha_{1i}\\
\vdots\\
\alpha_{ni}
\end{bmatrix}
=\begin{bmatrix}
| & | & & |\\
\vv_1 & \vv_2 & \cdots & \vv_n\\
| & | & & |
\end{bmatrix}
[\vv_i']_\sV.
\]
If we do the same for each vector in the basis $\sV'$, we have
\begin{equation}
  \label{eq:2}
\begin{bmatrix}
| & | & & |\\
\vv_1' & \vv_2' & \cdots & \vv_n'\\
| & | & & |
\end{bmatrix}
=
\begin{bmatrix}
| & | & & |\\
\vv_1 & \vv_2 & \cdots & \vv_n\\
| & | & & |
\end{bmatrix}
\begin{bmatrix}
\alpha_{11} & \alpha_{12}    & \cdots&  \alpha_{1n}\\
\alpha_{21} & \alpha_{22}    &       &      \\
\vdots     &               &\ddots &       \\
\alpha_{n1} &               &       &  \alpha_{nn}
\end{bmatrix}.
\end{equation}
The \emph{change-of-basis matrix} $P$ is the matrix on the right-hand side of
equation~(\ref{eq:2}).  That is, $P$ is formed by putting 
$[\vv_1']_\sV$ in the first column, $[\vv_2']_\sV$ in the second
column, and so on.  So 
\[
P = 
%\left[\begin{array}{cccc}
\begin{bmatrix}
| & | & & |\\
[\vv_1']_\sV & [\vv_2']_\sV & \cdots & [\vv_n']_\sV\\
| & | & & |
%  \end{array}\right] = 
\end{bmatrix} =
\begin{bmatrix}
\alpha_{11} & \alpha_{12}    & \cdots&  \alpha_{1n}\\
\alpha_{21} & \alpha_{22}    &       &      \\
\vdots     &               &\ddots &       \\
\alpha_{n1} &               &       &  \alpha_{nn}
\end{bmatrix}.
\]
If we define the matrices
\begin{equation}
  \label{eq:4}
B := \begin{bmatrix}
| & | & & |\\
\vv_1 & \vv_2 & \cdots & \vv_n\\
| & | & & |
\end{bmatrix}\quad \text{ and } \quad 
B' := \begin{bmatrix}
| & | & & |\\
\vv_1' & \vv_2' & \cdots & \vv_n'\\
| & | & & |
\end{bmatrix},
\end{equation}
then Equation~(\ref{eq:2}) becomes $B' = BP$.  Since $\sV$ is a basis, 
$B$ is invertible, so we see that $P$ can be computed as $P= B^{-1}B'$. 

\begin{theorem}
  Multiplication by the matrix $P$ converts the $\sV'$-basis representation of a
  vector into the $\sV$-basis representation of the same vector.
\end{theorem}
\begin{proof}
%% Suppose $[\vx]_\sV = (c_1, c_2, \dots, c_n)$
%% and $[\vx]_{\sV'} = (d_1, d_2, \dots, d_n)$,
%% so that
%% \[
%% c_1 \vv_1 + c_2 \vv_2 + \cdots +c_n \vv_n= \vx = d_1 \vv_1' + d_2 \vv_2' + \cdots +d_n \vv_n'.
%% \]
The claim is that $[\vx]_\sV = P [\vx]_{\sV'}$. Indeed, if we define the
matrices $B$ and $B'$ as in~(\ref{eq:4}), then we have
%% \[
%% B := \begin{bmatrix}
%% | & | & & |\\
%% \vv_1 & \vv_2 & \cdots & \vv_n\\
%% | & | & & |
%% \end{bmatrix}\quad \text{ and } \quad 
%% B' := \begin{bmatrix}
%% | & | & & |\\
%% \vv_1' & \vv_2' & \cdots & \vv_n'\\
%% | & | & & |
%% \end{bmatrix},
%% \]
%% then
$\vx = B[\vx]_\sV$ and $\vx = B'[\vx]_{\sV'}$. Also,
equation (\ref{eq:2}) can be written as $B' = BP$, so we have
\[
B P 
[\vx]_{\sV'}
%% \begin{bmatrix}
%% d_1\\
%% \vdots\\
%% d_n
%% \end{bmatrix}
=
B' [\vx]_{\sV'}
%% \begin{bmatrix}
%% d_1\\
%% \vdots\\
%% d_n
%% \end{bmatrix}
=\vx = 
%% \begin{bmatrix}
%% | & | & & |\\
%% \vv_1 & \vv_2 & \cdots & \vv_n\\
%% | & | & & |
%% \end{bmatrix}
B [\vx]_\sV.
%% \begin{bmatrix}
%% c_1\\
%% \vdots\\
%% c_n
%% \end{bmatrix}
\]
Now, since $B$ is clearly invertible (because $\sV$ is a basis), we can apply
$B^{-1}$ to both sides of the
equation $B P [\vx]_{\sV'} = B [\vx]_\sV$ to get the desired result.
\end{proof}
\section{Changing basis representations of linear transformations}
Let $V$ and $W$  be linear transformations and suppose $T : V \to W$ is a linear 
transformation, so that $T\vx \in W$ for each $\vx\in V$.
Assume we have the following bases:
\[
\sV = \{\vv_1, \vv_2, \dots, \vv_n\} \quad \text{ and }
\quad
\sV' = \{\vv_1', \vv_2', \dots, \vv_n'\} \quad \text{ (two bases for $V$)}
\]
\[
\sW = \{\vw_1, \vw_2, \dots, \vw_n\} \quad \text{ and }
\quad
\sW' = \{\vw_1', \vw_2', \dots, \vw_n'\} \quad \text{ (two bases for $W$)}
\]
Suppose $P$ is the change-of-basis matrix from $\sV'$ to $\sV$ of the last
section.  Suppose $Q$ is the change-of-basis matrix from $\sW'$ to $\sW$.

The $\sV,\sW$-basis representation of $T$ is a matrix, denoted by
$[T]_{\sV,\sW}$, that when applied to $[\vx]_\sV$ computes the value of
$T\vx$ with respect to the basis $\sW$.  More precisely, if we have a
vector $\vx\in V$, then we can apply the matrix $[T]_{\sV,\sW}$ to the
$\sV$-basis representation $[\vx]_{\sV}$ and obtain 
the $\sW$-basis representation of $T\vx$.  In other terms,
\[
[T\vx]_{\sW} = [T]_{\sV,\sW}[\vx]_\sV.
\] 
Suppose we are given the matrix $[T]_{\sV,\sW}$ and we wish to find the matrix 
$[T]_{\sV',\sW'}$.  The matrix $[T]_{\sV',\sW'}$ should give $[T\vx]_{\sW'}$
when applied to the vector $[\vx]_{\sV'}$, that is,
\[
[T\vx]_{\sW'} = [T]_{\sV',\sW'} [\vx]_{\sV'}.
\]
In order to use $[T]_{\sV,\sW}$ to effect this transformation, think about what
must happen to $[\vx]_{\sV'}$ before we can apply $[T]_{\sV,\sW}$ to it.  We
must first apply the matrix $P$ from the previous section to get 
$[\vx]_{\sV} = P[\vx]_{\sV'}$, the $\sV$-basis representation of $\vx$.  Only
then can we apply the original matrix $[T]_{\sV,\sW}$.  But the
result of this is the $\sW$-basis representation of $T\vx$; that is, 
\[
[T]_{\sV,\sW}P[\vx]_{\sV'} = [T]_{\sV,\sW}[\vx]_{\sV} = [T\vx]_{\sW},
\]
whereas we initially set out to compute $[T\vx]_{\sW'}$.  
So we simply apply the change-of-basis matrix that goes from $\sW$ to $\sW'$,
which is the inverse of $Q$.  We now arrive at the final result:
\[
[T\vx]_{\sW'} = [T]_{\sV',\sW'} [\vx]_{\sV'} = 
(Q^{-1}[T]_{\sV,\sW}P)[\vx]_{\sV'}.
\]
Since this holds for an arbitrary vector $\vx \in V$, we see that 
\begin{equation}
  \label{eq:3}
[T]_{\sV',\sW'} = Q^{-1}[T]_{\sV,\sW}P,
\end{equation}
and equation~(\ref{eq:3}) is the \emph{change-of-basis formula}.


To be clear, the matrices $P$ and $Q$ that appear in 
formula~(\ref{eq:3}) are
\[
P = 
\begin{bmatrix}
| & | & & |\\
[\vv_1']_\sV & [\vv_2']_\sV & \cdots & [\vv_n']_\sV\\
| & | & & |
\end{bmatrix}
\quad \text{ and } \quad 
Q = 
\begin{bmatrix}
| & | & & |\\
[\vw_1']_\sW & [\vw_2']_\sW & \cdots & [\vw_n']_\sW\\
| & | & & |
\end{bmatrix}.
\]


\end{document}
