%%This is the LaTeX file for the Math 317 Homework.  

%% You may use this file to fill in your solutions if you know how 
%% to compile a LaTeX file to turn it into a pdf document.  If you 
%% don't know how to do this but want to learn, there are plenty of
%% resources for learning about LaTeX on the net.  You may also ask
%% your professor for help.
%% 
%% For more information about using this file to complete the homework,
%% please see the NOTES section below of the comments below.

\documentclass[fleqn,11pt]{paper}
\usepackage[
letterpaper,
top    = 3cm,
bottom = 3cm,
left   = 3.00cm,
right  = 3.00cm]{geometry}

\usepackage{tikz-cd}
\usepackage{amsthm}
\usepackage{scalefnt}

%%%%%%%%%%%%%%%%%%%%%%%%%%%%%%%%%%%%%%%%
% Basic packages
%%%%%%%%%%%%%%%%%%%%%%%%%%%%%%%%%%%%%%%%
\usepackage{amsmath,amsthm,amssymb}
\usepackage[mathcal]{euscript}
\usepackage{mathtools}
\usepackage{etoolbox}
\usepackage{fancyhdr}
 \usepackage{xcolor}
\usepackage[colorlinks=true,urlcolor=blue,linkcolor=blue,citecolor=blue]{hyperref}
\usepackage{xspace}
\usepackage{comment}
\usepackage{url} % for url in bib entries
\usepackage{mathrsfs}

\theoremstyle{remark}
\newtheorem{theorem}{Theorem}
\newtheorem*{prop}{Proposition}
\newtheorem{problem}{Problem}
\newtheorem*{prob}{Problem}
\newtheorem*{solution}{{\bf Solution}}
\newtheorem*{hint}{{\it Hint}}
\newtheorem*{ex}{Exercise}


%%%%%%%%%%%%%%%%%%%%%%%%%%%%%%%%%%%%%%%%%%%%%%%%%%
%% Surround the problem and solution with 
%% \begin{ProbBox}  and   \end{ProbBox}
%% to prevent pagebreaks.
\newenvironment{ProbBox}{\noindent\begin{minipage}{\linewidth}}{\end{minipage}}

%%%%%%%%%%%%%%%%
% Acronyms     %
%%%%%%%%%%%%%%%%
\usepackage[acronym, shortcuts]{glossaries}

%% HERE IS HOW YOU DEFINE ACRONYMS:
\newacronym{FTA}{FTA}{Fundamental Theorem of Algebra}
\newacronym{CRT}{CRT}{Chinese Remainder Theorem}

% Make \ac robust.
\robustify{\ac}

%%%%%%%%%%%%%%%%%%%%%%%%
% Fancy page style     %
%%%%%%%%%%%%%%%%%%%%%%%%
\pagestyle{fancy}
\newcommand{\metadata}[2]{
  \lhead{}
  \chead{}
  \rhead{\bfseries Math 317: Linear Algebra}
  \lfoot{#1}
  \cfoot{#2}
  \rfoot{\thepage}
}
\renewcommand{\headrulewidth}{0.4pt}
\renewcommand{\footrulewidth}{0.4pt}


\newrobustcmd*{\vocab}[1]{\emph{#1}}
\newrobustcmd*{\latin}[1]{\textit{#1}}

%%%%%%%%%%%%%%%%%%%%%%%%%%%%%%%%%%
% Customize list enviroonments   %
%%%%%%%%%%%%%%%%%%%%%%%%%%%%%%%%%%
% package to customize three basic list environments: enumerate, itemize and description.
%% \usepackage{enumitem}
%% \setitemize{noitemsep, topsep=0pt, leftmargin=*}
%% \setenumerate{noitemsep, topsep=0pt, leftmargin=*}
%% \setdescription{noitemsep, topsep=0pt, leftmargin=*}

\usepackage{enumerate}
\usepackage{multicol}

%%%%%%%%%%%%%%%%%%%%%%%%%%%%
%% Space between problems  %
%%%%%%%%%%%%%%%%%%%%%%%%%%%%
\newrobustcmd*{\probskip}{\vskip1cm}


%%%%%%%%%%%%%%%%%%%%%%%%%%
%%    Math shortcuts     %
%%%%%%%%%%%%%%%%%%%%%%%%%%
%\newcommand\iff{\ensuremath{\Longleftrightarrow}}
\newcommand\join{\ensuremath{\vee}}
\newcommand\meet{\ensuremath{\wedge}}
\newcommand\R{\fld{R}}
%\newcommand\det{\ensuremath{\operatorname{det}}}
\newcommand\proj{\ensuremath{\operatorname{proj}}}
\newcommand\End{\ensuremath{\operatorname{End}}}
\newcommand\Aut{\ensuremath{\operatorname{Aut}}}
\newcommand\Hom{\ensuremath{\operatorname{Hom}}}
\newcommand{\Aff}{\ensuremath{\operatorname{Aff}}}
\newcommand{\ann}[1]{\ensuremath{\operatorname{ann}(#1)}}
\newcommand{\id}{\ensuremath{\operatorname{id}}}
\newcommand{\nulity}[1]{\ensuremath{\operatorname{null}(#1)}}
\renewcommand{\ker}[1]{\ensuremath{\operatorname{ker}(#1)}}
\renewcommand{\dim}[1]{\ensuremath{\operatorname{dim}(#1)}}
\newcommand\im[1]{\ensuremath{\operatorname{im}(#1)}}
\newcommand{\rank}[1]{\ensuremath{\operatorname{rank}(#1)}}
\newcommand{\trace}[1]{\ensuremath{\operatorname{trace}(#1)}}
\renewcommand{\phi}{\ensuremath{\varphi}}

\renewcommand{\vec}[1]{\mathbf{#1}}


%%%% NOTES %%%%%%%%%%%%%%%%%%%%%%%%%%%%%%%%%%%%%%%%%%%%%%%%%%%%%%%%% 
%%
%%    1. Write your answers inside a 
%%
%%               \begin{solution}...\end{solution}
%%
%%       environment.
%%
%%    2. After typing your answers into this Homework*.tex source file, you
%%       will have to compile it into a pdf document.  There are a number of ways to
%%       do that. Probably the easiest is to use the website called ShareLaTeX.com.
%%       Alternatively,
%%
%%           Linux: most come with TeX; otherwise do a full install of TeXLive.
%%           Mac OS X: you might try MacTeX. 
%%           Windows: try proTeXt maybe? (or switch to a better operating system) 
%%
%%       There is a Makefile in this directory, so on Linux you could just 
%%       enter `make` to compile all the Homework*.tex files at once.
%%
%%    3. Please don't hesitate to inform the professor if you have trouble; 
%%       post to Piazza or ask in lecture.
%%
%%    4. Try to use standard notation, as used in class and in the textbook.
%%       For the most basic symbols, you may wish to use LaTeX macros to keep 
%%       the conventions you use consistent and easy to remember.
%%       For example, to denote an algebra,
\newcommand\alg[1]{\ensuremath{\mathbf{#1}}}
\newcommand{\<}{\ensuremath{\langle}}
\renewcommand{\>}{\ensuremath{\rangle}}
%%       So, an algebra in LaTeX is typed as $\alg{A} = \<A, F\>$.
%%       Similarly, for a field, let's use:
\newcommand\fld[1]{\ensuremath{\mathbb{#1}}}
%%       So, a field in LaTeX is typed as $\fld{F}$.
%%
%%       To make a boldface vector, use backslash v in front of the 
%%       letter and add a new command for that letter here:
\newcommand\va{\vec{a}}
\newcommand\vb{\vec{b}}
\newcommand\vu{\vec{u}}
\newcommand\vv{\vec{v}}
\newcommand\vw{\vec{w}}
\newcommand\vx{\vec{x}}
\newcommand\vy{\vec{y}}
\newcommand\vz{\vec{z}}
\newcommand\vzero{\vec{0}}
\newcommand\sP{\ensuremath{\mathscr P}}
\newcommand\Span{\ensuremath{\operatorname{Span}}}
\newcommand\bR{\ensuremath{\mathbf R}}
\newcommand\bN{\ensuremath{\mathbf N}}
\newcommand\bC{\ensuremath{\mathbf C}}
%%
%%    5. Insert your name here!!!
%%
%%
\metadata       {Name:                     }{HW 6a (due 2016/03/04)}
%% Put your name here ^^^^^^^^^^^^^^^^^^^^^, like this:
%%
\author         {NAME:                     }
%% Put your name here ^^^^^^^^^^^^^^^^^^^^^, like this:
%%     \author{NAME: William DeMeo}
%%
%%    6. Update the title and date if necessary.
\title{Math 317: Homework 6b}
\date{Due: 4 Mar 2016}
%%
%%%%%%%%%%%%%%%%%%%%%%%%%%%%%%%%%%%%%%%%%%%%%%%%%%%%%%%%%%%%%%%%%%% 

\begin{document}

\maketitle

\section*{Section 3.2}

%--  PROBLEM 1  ---------------------------------------------------
\begin{problem}[SA 3.2.1]
  Show that if $B$ is obtained from $A$ by performing one or more elementary
  row operations, then $\bR(B) = \bR(A)$.
\end{problem}

\newpage
%--  PROBLEM 2  ---------------------------------------------------
\begin{problem}[SA 3.2.5]
  Suppose $A = LU$, where
  \[
  L =
  \begin{bmatrix*}[r]
    1& 0& 0\\
    2 &1& 0\\
    1& -1& 1
  \end{bmatrix*}
  \quad  \text{ and } \quad U =
  \begin{bmatrix*}[r]
    1 &2& 1& 1\\
    0& 0& 2& -2\\
    0& 0& 0& 0
  \end{bmatrix*}.\]
  Give vectors that span $\bR(A)$, $\bC(A)$, and $\bN(A)$.
\end{problem}
%% \medskip
%% \begin{solution}
%% %% type your answer here and uncomment these lines
%% \end{solution}
\newpage

%--  PROBLEM 3  ---------------------------------------------------
\begin{problem}[SA 3.2.6a]
Construct a matrix whose column space contains 
$\begin{bmatrix*}[r] 1\\1\\1 \end{bmatrix*}$ and 
$\begin{bmatrix*}[r] 0\\1\\1 \end{bmatrix*}$ and 
whose null space contains 
$\begin{bmatrix*}[r] 1\\0\\1 \end{bmatrix*}$ and 
$\begin{bmatrix*}[r] 0\\1\\0 \end{bmatrix*}$, or
explain why no such matrix exists.
\end{problem}
%% \medskip
%% \begin{solution}
%% %% type your answer here and uncomment these lines
%% \end{solution}

\newpage

%--  PROBLEM 4  ---------------------------------------------------
\begin{problem}[SA 3.2.10]
Let $A$ be an $m \times n$ matrix and $B$ be an $n \times p$ matrix. Prove that
\begin{enumerate}[a.]
\item $\bN(B) \subseteq \bN(AB)$.
\item $\bC(AB) \subseteq \bC(A)$. [{\it Hint:} Use Proposition 2.1.]
\item $\bN(B) = \bN(AB)$ when $A$ is $n \times n$ and nonsingular. [{\it Hint:} See the box on p. 12.]
\item $\bC(AB) = \bC(A)$ when $B$ is $n \times n$ and nonsingular.
\end{enumerate}
\end{problem}
%% \medskip
%% \begin{solution}
%% %% type your answer here and uncomment these lines
%% \end{solution}

\newpage

%--  PROBLEM 5  ---------------------------------------------------
\begin{problem}[SA 3.2.11]
Let $A$ be an $m \times n$ matrix. Prove that $\bN(A^{\top} A) = \bN(A)$. \\
~[{\it Hint:} Use the previous problem (SA 3.2.10) and Exercise SA 2.5.15.]
\end{problem}
%% \medskip
%% \begin{solution}
%% %% type your answer here and uncomment these lines
%% \end{solution}

\newpage

%--  PROBLEM  ---------------------------------------------------
\noindent The following problem is recommended but will not be graded.
\begin{prob}[SA 3.2.13]
Let $A$ be an $m \times n$ matrix with the property that $A^2 = A$.
\begin{enumerate}[a.]
\item Prove that $\bC(A) = \{\vx \in \R^n : \vx = A\vx\}$.
\item Prove that $\bN(A) = \{\vx \in \R^n : \vx= \vu - A\vu \text{ for some } \vu \in \R^n \}$.
\item Prove that $\bC(A) \cap \bN(A) = \{\vzero\}$.
\item Prove that $\bC(A) + \bN(A) = \R^n$.
\end{enumerate}
\end{prob}



\end{document}



















