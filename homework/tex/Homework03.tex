%%This is the LaTeX file for the Math 317 Homework.  

%% You may use this file to fill in your solutions if you know how 
%% to compile a LaTeX file to turn it into a pdf document.  If you 
%% don't know how to do this but want to learn, there are plenty of
%% resources for learning about LaTeX on the net.  You may also ask
%% your professor for help.
%% 
%% For more information about using this file to complete the homework,
%% please see the NOTES section below of the comments below.

\documentclass[fleqn,11pt]{paper}


\usepackage[
letterpaper,
top    = 3cm,
bottom = 3cm,
left   = 3.00cm,
right  = 3.00cm]{geometry}

\usepackage{tikz-cd}
\usepackage{amsthm}
\usepackage{scalefnt}
\usepackage{multicol}

%%%%%%%%%%%%%%%%%%%%%%%%%%%%%%%%%%%%%%%%
% Basic packages
%%%%%%%%%%%%%%%%%%%%%%%%%%%%%%%%%%%%%%%%
\usepackage{amsmath,amsthm,amssymb}
\usepackage{mathtools}
\usepackage{etoolbox}
\usepackage{fancyhdr}
 \usepackage{xcolor}
\usepackage[colorlinks=true,urlcolor=blue,linkcolor=blue,citecolor=blue]{hyperref}
\usepackage{xspace}
\usepackage{comment}
\usepackage{url} % for url in bib entries
\usepackage{mathrsfs}

\theoremstyle{remark}
\newtheorem{theorem}{Theorem}
\newtheorem*{prop}{Proposition}
\newtheorem{problem}{Problem}
\newtheorem*{prob}{Problem}
\newtheorem*{solution}{{\bf Solution}}
\newtheorem*{hint}{{\it Hint}}
\newtheorem*{ex}{Exercise}


%%%%%%%%%%%%%%%%%%%%%%%%%%%%%%%%%%%%%%%%%%%%%%%%%%
%% Surround the problem and solution with 
%% \begin{ProbBox}  and   \end{ProbBox}
%% to prevent pagebreaks.
\newenvironment{ProbBox}{\noindent\begin{minipage}{\linewidth}}{\end{minipage}}

%%%%%%%%%%%%%%%%
% Acronyms     %
%%%%%%%%%%%%%%%%
\usepackage[acronym, shortcuts]{glossaries}

%% HERE IS HOW YOU DEFINE ACRONYMS:
\newacronym{FTA}{FTA}{Fundamental Theorem of Algebra}
\newacronym{CRT}{CRT}{Chinese Remainder Theorem}

% Make \ac robust.
\robustify{\ac}

%%%%%%%%%%%%%%%%%%%%%%%%
% Fancy page style     %
%%%%%%%%%%%%%%%%%%%%%%%%
\pagestyle{fancy}
\newcommand{\metadata}[2]{
  \lhead{}
  \chead{}
  \rhead{\bfseries Math 317: Linear Algebra}
  \lfoot{#1}
  \cfoot{#2}
  \rfoot{\thepage}
}
\renewcommand{\headrulewidth}{0.4pt}
\renewcommand{\footrulewidth}{0.4pt}


\newrobustcmd*{\vocab}[1]{\emph{#1}}
\newrobustcmd*{\latin}[1]{\textit{#1}}

%%%%%%%%%%%%%%%%%%%%%%%%%%%%%%%%%%
% Customize list enviroonments   %
%%%%%%%%%%%%%%%%%%%%%%%%%%%%%%%%%%
% package to customize three basic list environments: enumerate, itemize and description.
%% \usepackage{enumitem}
%% \setitemize{noitemsep, topsep=0pt, leftmargin=*}
%% \setenumerate{noitemsep, topsep=0pt, leftmargin=*}
%% \setdescription{noitemsep, topsep=0pt, leftmargin=*}

\usepackage{enumerate}

%%%%%%%%%%%%%%%%%%%%%%%%%%%%
%% Space between problems  %
%%%%%%%%%%%%%%%%%%%%%%%%%%%%
\newrobustcmd*{\probskip}{\vskip1cm}


%%%%%%%%%%%%%%%%%%%%%%%%%%
%%    Math shortcuts     %
%%%%%%%%%%%%%%%%%%%%%%%%%%
%\newcommand\iff{\ensuremath{\Longleftrightarrow}}
\newcommand\join{\ensuremath{\vee}}
\newcommand\meet{\ensuremath{\wedge}}
\newcommand\R{\mathbb{R}}
\newcommand\proj{\ensuremath{\operatorname{proj}}}
\newcommand\End{\ensuremath{\operatorname{End}}}
\newcommand\Aut{\ensuremath{\operatorname{Aut}}}
\newcommand\Hom{\ensuremath{\operatorname{Hom}}}
\newcommand{\Aff}{\ensuremath{\operatorname{Aff}}}
\newcommand{\ann}[1]{\ensuremath{\operatorname{ann}(#1)}}
\newcommand{\id}{\ensuremath{\operatorname{id}}}
\newcommand{\nulity}[1]{\ensuremath{\operatorname{null}(#1)}}
\renewcommand{\ker}[1]{\ensuremath{\operatorname{ker}(#1)}}
\renewcommand{\dim}[1]{\ensuremath{\operatorname{dim}(#1)}}
\newcommand\im[1]{\ensuremath{\operatorname{im}(#1)}}
\newcommand{\rank}[1]{\ensuremath{\operatorname{rank}(#1)}}
\newcommand{\trace}[1]{\ensuremath{\operatorname{trace}(#1)}}
\renewcommand{\phi}{\ensuremath{\varphi}}

\renewcommand{\vec}[1]{\mathbf{#1}}



%%%% NOTES %%%%%%%%%%%%%%%%%%%%%%%%%%%%%%%%%%%%%%%%%%%%%%%%%%%%%%%%% 
%%
%%    1. Write your answers inside a 
%%
%%               \begin{solution}...\end{solution}
%%
%%       environment.
%%
%%    2. Enter your answers into this Homework*.tex source file, then compile it 
%%       into a pdf document.  There are a number of ways to do that. Probably 
%%       the easiest is to use the website called ShareLaTeX.com. Alternatively,
%%
%%           Linux: most distros come with TeX, but you may want a full(er) version of TeXLive.
%%           Mac OS X: you might try MacTeX. 
%%           Windows: proTeXt maybe (or switch to a better operating system) 
%%
%%       There is a Makefile in this directory, so on Linux you could just 
%%       enter `make` to compile all the Homework*.tex files at once.
%%
%%    3. Please don't hesitate to inform the professor if you have trouble, or open
%%       a ``New issue'' on GitHub or post to Piazza or ask in lecture.  Otherwise,
%%       send an email to williamdemeo at gmail.
%%
%%    4. Try to use standard notation, as used in class and in the textbook.
%%       For the most basic symbols, you may wish to use LaTeX macros to keep 
%%       the conventions you use consistent and easy to remember.
%%
%%       For example, to make a boldface vector, use backslash v in front of the 
%%       letter and add a new command for that letter, as follows:
         \newcommand\va{\vec{a}}
         \newcommand\vb{\vec{b}}
         \newcommand\vc{\vec{c}}
         \newcommand\vu{\vec{u}}
         \newcommand\vv{\vec{v}}
         \newcommand\vw{\vec{w}}
         \newcommand\vx{\vec{x}}
         \newcommand\vy{\vec{y}}
         \newcommand\vz{\vec{z}}
         \newcommand\vzero{\vec{0}}
         \newcommand\sP{\ensuremath{\mathscr P}}
         \newcommand\Span{\ensuremath{\operatorname{Span}}}
%%
%%    5. Insert your name here!!!
%%

         \metadata       {Name:                     }{HW 3 (due 2016/02/05)}
         %% Put your name here ^^^^^^^^^^^^^^^^^^^^^, like this:
         %% \metadata  {Name: William DeMeo}{HW 3 (due 2015/09/21)}  
         %%
         
         \author         {NAME:                     }
         %% Put your name here ^^^^^^^^^^^^^^^^^^^^^, like this:
         %%     \author{NAME: William DeMeo}
         %%
%%
%%
%%    8. Update the title and date if necessary.
         \title{Math 317: Homework 3}
         \date{Due: 5 February 2016}
%%
%%%%%%%%%%%%%%%%%%%%%%%%%%%%%%%%%%%%%%%%%%%%%%%%%%%%%%%%%%%%%%%%%%% 


\begin{document}

\maketitle


%--  PROBLEM 1  ---------------------------------------------------
%% \begin{problem}[SA 1.5.1]
%%   By solving a system of equations, write the vector
%%   \[ \vb = \begin{bmatrix*}[r] 3 \\ 0 \\ -2 \end{bmatrix*}
%%   \; \text{ as a linear combination of the vectors } \;
%%   \vv_1 = \begin{bmatrix*}[r] 1 \\ 0 \\ -1 \end{bmatrix*}, \;
%%   \vv_2 = \begin{bmatrix*}[r] 0 \\ 1 \\ 2 \end{bmatrix*},
%%   \; \text{ and } \; \vv_3 = \begin{bmatrix*}[r] 2 \\ 1 \\ 1 \end{bmatrix*}. \]
%% (Do it by hand and show your work.  You can use Sage, but only to check that your answer is correct.)
%% \end{problem}

%% \newpage

%--  PROBLEM 2  ---------------------------------------------------
\begin{problem}[SA 1.5.2b]
  Is the vector
  \[
  \vb = \begin{bmatrix*}[r] 1 \\ -1 \\ 1 \\ -1\end{bmatrix*} \;
  \text{ a linear combination of the vectors } \;
  \vv_1 = \begin{bmatrix*}[r] 1 \\ 0 \\ 1 \\ -2\end{bmatrix*},
  \;
  \vv_2 = \begin{bmatrix*}[r] 0 \\ -1 \\ 0 \\ 1 \end{bmatrix*},
  \;
  \vv_3 = \begin{bmatrix*}[r] 1\\ -2 \\ 1 \\ 0 \end{bmatrix*}?
  \]
  If the answer is ``no,'' prove it.  If the answer is ``yes,'' write down the \emph{general 
  solution} to the system $A\vx = \vb$, where $A$ is the matrix with the vector $\vv_i$ in column $i$.
  (You must do this problem by hand and show your work to get credit.  You may use Sage to check your answer.)
\end{problem}

\newpage

%% %--  PROBLEM 3 (old) ---------------------------------------------------
%% \begin{problem}[SA 1.5.3a]
%%   Find constraint equations (if any) that $\vb$ must satisfy in order for
%%   $A\vx = \vb$ to be consistent, when \[A = \begin{bmatrix*}[r] 3 & -1\\ 6 & -2 \\ -9 & 3 \end{bmatrix*}.\]
%% \end{problem}

%--  PROBLEM 3 (new) ---------------------------------------------------
\begin{problem}[SA 1.5.3cd]
  Find constraint equations (if any) that $\vb$ must satisfy in order for
  $A\vx = \vb$ to be consistent. 
\begin{multicols}{2}
\begin{enumerate}
\item[{\bf c.}] \[A = \begin{bmatrix*}[r]0&1&1\\1&2&1\\2&1&-1\end{bmatrix*}\]
\item[{\bf d.}] \[A = \begin{bmatrix*}[r]1&1&2\\2&-1&2\\1&-2&1\end{bmatrix*}\]
\end{enumerate}
\end{multicols}
\noindent ({\it Hint:} See Examples 3, 4, and 5 of Section 1.5.)
\end{problem}

%% %--  PROBLEM 4  ---------------------------------------------------
%% \begin{problem}[SA 1.5.4c]
%%   Find constraint equations that $\vb$ must satisfy in order to be an element of
%%   $V = \Span\bigl((1,0,1,1), (0,1,1,2), (2,-1,1,0)\bigr)$.
%% \end{problem}

\newpage

%--  PROBLEM 5  ---------------------------------------------------
\begin{problem}[SA 1.5.10]
  Let $A$ be an $m \times n$ matrix. Prove, or disprove with counterexample,
  the following claim:
  If $A\vx = \vzero$ has only the trivial solution $\vx = \vzero$, then for each 
  $\vb\in \R^m$ the system $A\vx = \vb$ has a unique solution.
\end{problem}


\newpage

%--  PROBLEM 6  ---------------------------------------------------
\begin{problem}[SA 1.5.12cd]
  In each case, give positive integers $m$ and $n$ and an example of an
  $m \times n$ matrix $A$ with the stated property, or explain why none can exist.
  \begin{itemize}
  \item[c.] $A\vx = \vb$ has no solutions for some $\vb \in \R^m$ and one solution
    for every other $\vb\in \R^m$.
  \item[d.] $A\vx = \vb$ has infinitely many solutions for every $\vb \in \R^m$. 
  \end{itemize}
\end{problem}

\newpage

%% %--  PROBLEM 7  ---------------------------------------------------
%% \begin{problem}[SA 2.1.6ad]
%%   Assume $A$, $B$, and $C$ are $n\times n$ matrices and 
%%   prove, or disprove with a counterexample, the following claims:
%%   \begin{itemize}
%%   \item[a.] If $AB = CB$ and $B$ is not the zero matrix, then $A = C$.
%%   %\item[c.] $(A+B)(A-B) = A^2 - B^2$.
%%   \item[d.] If $AB = CB$ and $B$ is nonsingular, then $A = C$.

%%   \end{itemize}
%% \end{problem}

%% \newpage

%--  PROBLEM 8  ---------------------------------------------------
\begin{problem}[cf.~SA 2.1.8]
Consider the following matrix 
  \[
  A = \begin{bmatrix*}[r]
    d_1     & 0      & \cdots & 0      \\
    0       & d_2    & \ddots & \vdots \\
    \vdots  & \ddots &\ddots&    0      \\
    0       & \cdots &    0  & d_n
  \end{bmatrix*}.
  \]
\begin{enumerate}[{\bf a.}]
\item 
Find a formula for $A^k$ that holds for positive integers $k$. 
In other words, express the product $AA\cdots A$ (of $k$ factors)
as a function of the entries of the matrix $A$.\\
({\it Hint:} if you're having trouble starting this one, try computing 
higher powers of a small example, like $A = \begin{bmatrix*}[r] 2 & 0\\ 0 & 3\end{bmatrix*}$,
then notice the pattern and guess the formula.)
\item Use the principle of induction to prove that your formula is correct.
\end{enumerate}
\end{problem}

\newpage

%% %--  PROBLEM 9  ---------------------------------------------------
%% \begin{problem}[SA 2.1.10]
%%   Suppose $A$ and $B$ are nonsingular $n \times n$ matrices.
%%   Prove that $AB$ is nonsingular.

%%   \begin{quote}
%%   {\it Hint:} Although it is tempting to try to show that the reduced echelon
%%   form of AB is the identity matrix, there is no direct way to do this. As is
%%   the case in most non-numerical problems regarding nonsingularity, you should
%%   remember that $AB$ is nonsingular precisely when the only solution of
%%   $(AB)\vx = \vzero$ is $\vx = \vzero$.
%%   \end{quote}
  
%% \end{problem}


%% \newpage

%--  PROBLEM 10  ---------------------------------------------------
\begin{problem}[SA 2.1.11a]
  Suppose $A \in \R^{m\times n}$,
  $B \in \R^{n\times m}$, and $BA = I_n$.
  Prove that if for some $\vb \in \R^m$ the
  equation $A\vx = \vb$ has a solution, then that solution is unique.
  \begin{quote}
    {\it Hint:}
    To prove the statement ``if a solution exists, then it is unique,'' one approach (which works well
    here) is to suppose that $\vx$ satisfies the equation and find a formula
    that determines it. Another approach is to assume that $\vx$ and $\vy$ are
    both solutions and then use the equations to prove that $\vx = \vy$.
  \end{quote}
  
\end{problem}


%% \bibliographystyle{plain}
%% \bibliography{refs}

\end{document}
