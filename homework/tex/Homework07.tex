%%This is the LaTeX file for the Math 317 Homework.  

%% You may use this file to fill in your solutions if you know how 
%% to compile a LaTeX file to turn it into a pdf document.  If you 
%% don't know how to do this but want to learn, there are plenty of
%% resources for learning about LaTeX on the net.  You may also ask
%% your professor for help.
%% 
%% For more information about using this file to complete the homework,
%% please see the NOTES section below of the comments below.

\documentclass[fleqn,11pt]{paper}


\usepackage[
letterpaper,
top    = 3cm,
bottom = 3cm,
left   = 3.00cm,
right  = 3.00cm]{geometry}

\usepackage{tikz-cd}
\usepackage{amsthm}
\usepackage{scalefnt}

%%%%%%%%%%%%%%%%%%%%%%%%%%%%%%%%%%%%%%%%
% Basic packages
%%%%%%%%%%%%%%%%%%%%%%%%%%%%%%%%%%%%%%%%
\usepackage{amsmath,amsthm,amssymb}
\usepackage{mathtools}
\usepackage{etoolbox}
\usepackage{fancyhdr}
\usepackage{xcolor}
\usepackage[colorlinks=true,urlcolor=blue,linkcolor=blue,citecolor=blue]{hyperref}
\usepackage{xspace}
\usepackage{comment}
\usepackage{url} % for url in bib entries
\usepackage{mathrsfs}

\theoremstyle{remark}
\newtheorem{theorem}{Theorem}
\newtheorem*{prop}{Proposition}
\newtheorem{problem}{Problem}
\newtheorem*{prob}{Problem}
\newtheorem*{solution}{{\bf Solution}}
\newtheorem*{hint}{{\it Hint}}
\newtheorem*{ex}{Exercise}


%%%%%%%%%%%%%%%%%%%%%%%%%%%%%%%%%%%%%%%%%%%%%%%%%%
%% Surround the problem and solution with 
%% \begin{ProbBox}  and   \end{ProbBox}
%% to prevent pagebreaks.
\newenvironment{ProbBox}{\noindent\begin{minipage}{\linewidth}}{\end{minipage}}

%%%%%%%%%%%%%%%%
% Acronyms     %
%%%%%%%%%%%%%%%%
\usepackage[acronym, shortcuts]{glossaries}

%% HERE IS HOW YOU DEFINE ACRONYMS:
\newacronym{FTA}{FTA}{Fundamental Theorem of Algebra}
\newacronym{CRT}{CRT}{Chinese Remainder Theorem}

% Make \ac robust.
\robustify{\ac}

%%%%%%%%%%%%%%%%%%%%%%%%
% Fancy page style     %
%%%%%%%%%%%%%%%%%%%%%%%%
\pagestyle{fancy}
\newcommand{\metadata}[2]{
  \lhead{}
  \chead{}
  \rhead{\bfseries Math 317: Linear Algebra}
  \lfoot{#1}
  \cfoot{#2}
  \rfoot{\thepage}
}
\renewcommand{\headrulewidth}{0.4pt}
\renewcommand{\footrulewidth}{0.4pt}


\newrobustcmd*{\vocab}[1]{\emph{#1}}
\newrobustcmd*{\latin}[1]{\textit{#1}}

%%%%%%%%%%%%%%%%%%%%%%%%%%%%%%%%%%
% Customize list enviroonments   %
%%%%%%%%%%%%%%%%%%%%%%%%%%%%%%%%%%
% package to customize three basic list environments: enumerate, itemize and description.
%% \usepackage{enumitem}
%% \setitemize{noitemsep, topsep=0pt, leftmargin=*}
%% \setenumerate{noitemsep, topsep=0pt, leftmargin=*}
%% \setdescription{noitemsep, topsep=0pt, leftmargin=*}

\usepackage{enumerate}

%%%%%%%%%%%%%%%%%%%%%%%%%%%%
%% Space between problems  %
%%%%%%%%%%%%%%%%%%%%%%%%%%%%
\newrobustcmd*{\probskip}{\vskip1cm}


%%%%%%%%%%%%%%%%%%%%%%%%%%
%%    Math shortcuts     %
%%%%%%%%%%%%%%%%%%%%%%%%%%
%\newcommand\iff{\ensuremath{\Longleftrightarrow}}
\newcommand\join{\ensuremath{\vee}}
\newcommand\meet{\ensuremath{\wedge}}
\newcommand\R{\ensuremath{\mathbb{R}}}
\newcommand\proj{\ensuremath{\operatorname{proj}}}
\newcommand\End{\ensuremath{\operatorname{End}}}
\newcommand\Aut{\ensuremath{\operatorname{Aut}}}
\newcommand\Hom{\ensuremath{\operatorname{Hom}}}
\newcommand{\Aff}{\ensuremath{\operatorname{Aff}}}
\newcommand{\id}{\ensuremath{\operatorname{id}}}
%% \newcommand{\ann}[1]{\ensuremath{\operatorname{ann}(#1)}}
%% \newcommand{\nullity}[1]{\ensuremath{\operatorname{null}(#1)}}
%% \renewcommand{\ker}[1]{\ensuremath{\operatorname{ker}(#1)}}
%% \renewcommand{\dim}[1]{\ensuremath{\operatorname{dim}(#1)}}
%% \newcommand\im[1]{\ensuremath{\operatorname{im}(#1)}}
%% \newcommand{\rank}[1]{\ensuremath{\operatorname{rank}(#1)}}
%% \newcommand{\trace}[1]{\ensuremath{\operatorname{trace}(#1)}}
\newcommand{\ann}{\ensuremath{\operatorname{ann}}}
\newcommand{\nullity}{\ensuremath{\operatorname{null}}}
\renewcommand{\ker}{\ensuremath{\operatorname{ker}}}
\renewcommand{\dim}{\ensuremath{\operatorname{dim}}}
\newcommand\im{\ensuremath{\operatorname{im}}}
\newcommand{\rank}{\ensuremath{\operatorname{rank}}}
\newcommand{\trace}{\ensuremath{\operatorname{trace}}}
\renewcommand{\phi}{\ensuremath{\varphi}}
\newcommand\bA{\ensuremath{\mathbf A}}
\newcommand\bB{\ensuremath{\mathbf B}}
\newcommand\bC{\ensuremath{\mathbf C}}
\newcommand\bN{\ensuremath{\mathbf N}}
\newcommand\bR{\ensuremath{\mathbf R}}
\newcommand\sP{\ensuremath{\mathcal P}}
\newcommand\Span{\ensuremath{\operatorname{Span}}}

%%%% NOTES %%%%%%%%%%%%%%%%%%%%%%%%%%%%%%%%%%%%%%%%%%%%%%%%%%%%%%%%% 
%%
%%    1. Type your answers inside a 
%%
%%               \begin{solution}...\end{solution}
%%
%%       environment.
%%
%%    2. Try to use standard notation, as used in class and in the textbook.
%%       For the most basic symbols, you may wish to use LaTeX macros to keep 
%%       the conventions you use consistent and easy to remember.
%%
%%       To make a boldface vector, use backslash v in front of the 
%%       letter and add a new command for that letter here:
         \renewcommand{\vec}[1]{\mathbf{#1}}
         \newcommand\va{\vec{a}}
         \newcommand\vb{\vec{b}}
         \newcommand\vu{\vec{u}}
         \newcommand\vv{\vec{v}}
         \newcommand\vw{\vec{w}}
         \newcommand\vx{\vec{x}}
         \newcommand\vy{\vec{y}}
         \newcommand\vz{\vec{z}}
         \newcommand\vzero{\vec{0}}

         \metadata       {HW 7}{Due: 2016/03/11}
         
%%
%%    3. Insert your name here!!!
%%
         \author         {NAME:                     }
         %% Put your name here ^^^^^^^^^^^^^^^^^^^^^, like this:
         %%     \author{NAME: William DeMeo}
         %%
%%
%%
%%    4. Update the title and date if necessary.
         \title{Math 317: Homework 7 \\{\large Due: 11 March 2016}}
         \date{Due: 11 March 2016}
%%
%%%%%%%%%%%%%%%%%%%%%%%%%%%%%%%%%%%%%%%%%%%%%%%%%%%%%%%%%%%%%%%%%%% 

\begin{document}

\maketitle

\section*{Section 3.3}
%--  PROBLEM 1  ---------------------------------------------------
\begin{problem}[SA 3.3.3]
Decide whether the following sets of vectors give a basis for the indicated
space. (Show your work and/or justify your answer.)
\begin{enumerate}[a.]
\item $\{(1, 2, 1), (2, 4, 5), (1, 2, 3)\}$; $\R^3$.
\item $\{(1, 0, 1), (1, 2, 4), (2, 2, 5), (2, 2, -1)\}$; $\R^3$.
\item $\{(1, 0, 2, 3), (0, 1, 1, 1), (1, 1, 4, 4)\}$; $\R^4$.
\item $\{(1, 0, 2, 3), (0, 1, 1, 1), (1, 1, 4, 4), (2, -2, 1, 2)\}$; $\R^4$.
\end{enumerate}
\end{problem}
%% \medskip
%% \begin{solution}
%%  (type answer here; uncomment these and surrounding lines)
%% \end{solution}

\newpage

%--  PROBLEM 2  ---------------------------------------------------
\begin{problem}[SA 3.3.4ac]
In each case, check that $\{\vv_1 ,\dots , \vv_n \}$ 
is a basis for $\R^n$ and give the coordinates of the
given vector $\vb \in \R^n$ with respect to that basis.
\begin{enumerate}[a.]
\item 
$\vv_1 =\begin{bmatrix*}[r] 2\\3 \end{bmatrix*}$,
$\vv_2 =\begin{bmatrix*}[r] 3\\5 \end{bmatrix*}$,
$\vb =\begin{bmatrix*}[r] 3\\4 \end{bmatrix*}$.
\item[c.]
$\vv_1 =\begin{bmatrix*}[r] 1\\0\\1 \end{bmatrix*}$,
$\vv_2 =\begin{bmatrix*}[r] 1\\1\\2 \end{bmatrix*}$,
$\vv_3 =\begin{bmatrix*}[r] 1\\1\\1 \end{bmatrix*}$,
$\vb =\begin{bmatrix*}[r] 3\\0\\1 \end{bmatrix*}$.
\end{enumerate}
\end{problem}
%% \medskip
%% \begin{solution}
%% \end{solution}

\newpage

%--  PROBLEM 3  ---------------------------------------------------
\begin{problem}[SA 3.3.5b]
Give a basis for each of the subspaces 
$\bR(A)$, $\bC(A)$, $\bN(A)$
where\\[4pt]
$A = \begin{bmatrix*}[r] 1&1&0\\2&1&1\\1&-1&2\end{bmatrix*}$.
\end{problem}
%% \medskip
%% \begin{solution}
%% \end{solution}

\newpage

%--  PROBLEM 4  ---------------------------------------------------
\begin{problem}[SA 3.3.11]
Suppose $\vv_1,\dots, \vv_n$ are nonzero, mutually orthogonal vectors in $\R^n$.
\begin{enumerate}[a.]
\item 
Prove that they form a basis for $\R^n$.
\item 
Given any $\vx \in \R^n$, give an explicit formula for the coordinates of 
$\vx$ with respect to the basis $\{\vv_1,\dots, \vv_n\}$.
\item Deduce from your answer to part b that $\vx = \sum_{i=1}^n \proj_{\vv_i}\vx$.
\end{enumerate}
\end{problem}
%% \medskip
%% \begin{solution}
%% \end{solution}

\newpage

%--  PROBLEM 5  ---------------------------------------------------
\begin{problem}[SA 3.4.1bcd]
For each subspace $V$, find a basis and determine $\dim V$.
\begin{enumerate}
\item[b.]
  $V = \{\vx \in \R^4 : x_1 + x_2 + x_3 + x_4 = 0, \; x_ 2 + x_4 = 0\} \subseteq  \R^4$.
\item[c.] $V = \left(\Span\{(1, 2, 3)\}\right)^\bot \subseteq \R^3$.
\item[d.] $V = \{\vx \in \R^5 : x_1 = x_2, \; x_3 = x_4 \} \subseteq \R^5$.
\end{enumerate}
\end{problem}
%% \medskip
%% \begin{solution}
%% \end{solution}

\newpage

%--  PROBLEM 6  ---------------------------------------------------
\begin{problem} %[cf.~SA 3.4.3]
  Let
  $A = \begin{bmatrix*}[r]
      1&-2 & 1 & 0 \\
      2 & -4 & 3 & -1
      \end{bmatrix*}$.
  \begin{enumerate}[(a)]
  \item Give bases for $\bR(A)$, $\bN(A)$, $\bC(A)$, and $\bN(A^\top)$.
  \item Use your answer to Part (a) to determine the dimension
    of each of these subspaces; confirm your answer using
    the ``Nullity-Rank  Theorem'' (Corollary 4.7 of the text).
  \item Using your answer to Part (a), verify the orthogonality conditions
    given in Theorem 3.2.5. (That is, check $\bN(A)^\bot = \bR(A)$ and $\bN(A^\top)^\bot = \bC(A)$.)
  \end{enumerate}
\end{problem}
%% \medskip
%% \begin{solution}
%% \end{solution}

\newpage

%--  PROBLEM 7  ---------------------------------------------------
\begin{problem}[SA 3.4.24]
  Continuing Exercise 3.2.10 from last week's homework... \\[4pt]
  Let $A$ be an $m\times n$ matrix.
\begin{enumerate}[a.]
\item  Use Theorem 2.5 to prove that $\bN(A^\top A) = \bN(A)$.\\[4pt]
  ({\it Hint:} if $\vx \in \bN(A^\top A)$, then $A\vx \in \bC(A) \cap \bN(A^\top)$.)
\item Prove that $\rank(A) = \rank(A^\top A)$.
\item Prove that $\bC(A^\top A) = \bC(A^\top)$.
\end{enumerate}
\end{problem}
%% \medskip
%% \begin{solution}
%% \end{solution}
\newpage

%--  PROBLEM  ---------------------------------------------------
\noindent The following problem is recommended but not required; it will not be graded.
\begin{prob}[SA 3.4.17]
  Let $V \leq R^n$ be a subspace.
  Prove that any linearly independent set of vectors in $V$ can
  be extended to a basis for $V$. 
  In other words, prove the following: if $\dim V > k$ and if we are
  given a linearly independent set of vectors
  $S=\{\vv_1, \vv_2, \dots, \vv_k\} \subseteq V$, 
  then there exist vectors
  $\vv_{k+1}, \dots, \vv_\ell \in V$
  such that $\{\vv_1, \vv_2, \dots, \vv_\ell\}$ is a basis for $V$.
\end{prob}
%% \medskip
%% \begin{solution}
%% \end{solution}



%% \bibliographystyle{plain}
%% \bibliography{refs}

\end{document}



















