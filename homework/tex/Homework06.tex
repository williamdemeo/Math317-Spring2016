%%This is the LaTeX file for the Math 317 Homework.  

%% You may use this file to fill in your solutions if you know how 
%% to compile a LaTeX file to turn it into a pdf document.  If you 
%% don't know how to do this but want to learn, there are plenty of
%% resources for learning about LaTeX on the net.  You may also ask
%% your professor for help.
%% 
%% For more information about using this file to complete the homework,
%% please see the NOTES section below of the comments below.

\documentclass[fleqn,11pt]{paper}
\usepackage[
letterpaper,
top    = 2cm,
bottom = 2cm,
left   = 2.00cm,
right  = 2.00cm]{geometry}

\usepackage{tikz-cd}
\usepackage{amsthm}
\usepackage{scalefnt}

%%%%%%%%%%%%%%%%%%%%%%%%%%%%%%%%%%%%%%%%
% Basic packages
%%%%%%%%%%%%%%%%%%%%%%%%%%%%%%%%%%%%%%%%
\usepackage{amsmath,amsthm,amssymb}
\usepackage[mathcal]{euscript}
\usepackage{mathtools}
\usepackage{etoolbox}
\usepackage{fancyhdr}
 \usepackage{xcolor}
\usepackage[colorlinks=true,urlcolor=blue,linkcolor=blue,citecolor=blue]{hyperref}
\usepackage{xspace}
\usepackage{comment}
\usepackage{url} % for url in bib entries
\usepackage{mathrsfs}

\theoremstyle{remark}
\newtheorem{theorem}{Theorem}
\newtheorem*{prop}{Proposition}
\newtheorem{problem}{Problem}
\newtheorem*{prob}{Problem}
\newtheorem*{solution}{{\bf Solution}}
\newtheorem*{hint}{{\it Hint}}
\newtheorem*{ex}{Exercise}


%%%%%%%%%%%%%%%%%%%%%%%%%%%%%%%%%%%%%%%%%%%%%%%%%%
%% Surround the problem and solution with 
%% \begin{ProbBox}  and   \end{ProbBox}
%% to prevent pagebreaks.
\newenvironment{ProbBox}{\noindent\begin{minipage}{\linewidth}}{\end{minipage}}

%%%%%%%%%%%%%%%%
% Acronyms     %
%%%%%%%%%%%%%%%%
\usepackage[acronym, shortcuts]{glossaries}

%% HERE IS HOW YOU DEFINE ACRONYMS:
\newacronym{FTA}{FTA}{Fundamental Theorem of Algebra}
\newacronym{CRT}{CRT}{Chinese Remainder Theorem}

% Make \ac robust.
\robustify{\ac}

%%%%%%%%%%%%%%%%%%%%%%%%
% Fancy page style     %
%%%%%%%%%%%%%%%%%%%%%%%%
\pagestyle{fancy}
\newcommand{\metadata}[2]{
  \lhead{}
  \chead{}
  \rhead{\bfseries Math 317: Linear Algebra}
  \lfoot{#1}
  \cfoot{#2}
  \rfoot{\thepage}
}
\renewcommand{\headrulewidth}{0.4pt}
\renewcommand{\footrulewidth}{0.4pt}


\newrobustcmd*{\vocab}[1]{\emph{#1}}
\newrobustcmd*{\latin}[1]{\textit{#1}}

%%%%%%%%%%%%%%%%%%%%%%%%%%%%%%%%%%
% Customize list enviroonments   %
%%%%%%%%%%%%%%%%%%%%%%%%%%%%%%%%%%
% package to customize three basic list environments: enumerate, itemize and description.
%% \usepackage{enumitem}
%% \setitemize{noitemsep, topsep=0pt, leftmargin=*}
%% \setenumerate{noitemsep, topsep=0pt, leftmargin=*}
%% \setdescription{noitemsep, topsep=0pt, leftmargin=*}

\usepackage{enumerate}
\usepackage{multicol}

%%%%%%%%%%%%%%%%%%%%%%%%%%%%
%% Space between problems  %
%%%%%%%%%%%%%%%%%%%%%%%%%%%%
\newrobustcmd*{\probskip}{\vskip1cm}


%%%%%%%%%%%%%%%%%%%%%%%%%%
%%    Math shortcuts     %
%%%%%%%%%%%%%%%%%%%%%%%%%%
%\newcommand\iff{\ensuremath{\Longleftrightarrow}}
\newcommand\join{\ensuremath{\vee}}
\newcommand\meet{\ensuremath{\wedge}}
\newcommand\R{\fld{R}}
%\newcommand\det{\ensuremath{\operatorname{det}}}
\newcommand\proj{\ensuremath{\operatorname{proj}}}
\newcommand\End{\ensuremath{\operatorname{End}}}
\newcommand\Aut{\ensuremath{\operatorname{Aut}}}
\newcommand\Hom{\ensuremath{\operatorname{Hom}}}
\newcommand{\Aff}{\ensuremath{\operatorname{Aff}}}
\newcommand{\ann}[1]{\ensuremath{\operatorname{ann}(#1)}}
\newcommand{\id}{\ensuremath{\operatorname{id}}}
\newcommand{\nulity}[1]{\ensuremath{\operatorname{null}(#1)}}
\renewcommand{\ker}[1]{\ensuremath{\operatorname{ker}(#1)}}
\renewcommand{\dim}[1]{\ensuremath{\operatorname{dim}(#1)}}
\newcommand\im[1]{\ensuremath{\operatorname{im}(#1)}}
\newcommand{\rank}[1]{\ensuremath{\operatorname{rank}(#1)}}
\newcommand{\trace}[1]{\ensuremath{\operatorname{trace}(#1)}}
\renewcommand{\phi}{\ensuremath{\varphi}}

\renewcommand{\vec}[1]{\mathbf{#1}}


%%%% NOTES %%%%%%%%%%%%%%%%%%%%%%%%%%%%%%%%%%%%%%%%%%%%%%%%%%%%%%%%% 
%%
%%    1. Write your answers inside a 
%%
%%               \begin{solution}...\end{solution}
%%
%%       environment.
%%
%%    2. After typing your answers into this Homework*.tex source file, you
%%       will have to compile it into a pdf document.  There are a number of ways to
%%       do that. Probably the easiest is to use the website called ShareLaTeX.com.
%%       Alternatively,
%%
%%           Linux: most come with TeX; otherwise do a full install of TeXLive.
%%           Mac OS X: you might try MacTeX. 
%%           Windows: try proTeXt maybe? (or switch to a better operating system) 
%%
%%       There is a Makefile in this directory, so on Linux you could just 
%%       enter `make` to compile all the Homework*.tex files at once.
%%
%%    3. Please don't hesitate to inform the professor if you have trouble; 
%%       post to Piazza or ask in lecture.
%%
%%    4. Try to use standard notation, as used in class and in the textbook.
%%       For the most basic symbols, you may wish to use LaTeX macros to keep 
%%       the conventions you use consistent and easy to remember.
%%       For example, to denote an algebra,
\newcommand\alg[1]{\ensuremath{\mathbf{#1}}}
\newcommand{\<}{\ensuremath{\langle}}
\renewcommand{\>}{\ensuremath{\rangle}}
%%       So, an algebra in LaTeX is typed as $\alg{A} = \<A, F\>$.
%%       Similarly, for a field, let's use:
\newcommand\fld[1]{\ensuremath{\mathbb{#1}}}
%%       So, a field in LaTeX is typed as $\fld{F}$.
%%
%%       To make a boldface vector, use backslash v in front of the 
%%       letter and add a new command for that letter here:
\newcommand\va{\vec{a}}
\newcommand\vb{\vec{b}}
\newcommand\vu{\vec{u}}
\newcommand\vv{\vec{v}}
\newcommand\vw{\vec{w}}
\newcommand\vx{\vec{x}}
\newcommand\vy{\vec{y}}
\newcommand\vz{\vec{z}}
\newcommand\vzero{\vec{0}}
\newcommand\sP{\ensuremath{\mathscr P}}
\newcommand\Span{\ensuremath{\operatorname{Span}}}
%%
%%    5. Insert your name here!!!
%%
%%
\metadata       {Name:                     }{HW 6 (due 2016/02/26)}
%% Put your name here ^^^^^^^^^^^^^^^^^^^^^, like this:
%%
\author         {NAME:                     }
%% Put your name here ^^^^^^^^^^^^^^^^^^^^^, like this:
%%     \author{NAME: William DeMeo}
%%
%%    6. Update the title and date if necessary.
\title{Math 317: Homework 6}
\date{Due: 26 Feb 2016}
%%
%%%%%%%%%%%%%%%%%%%%%%%%%%%%%%%%%%%%%%%%%%%%%%%%%%%%%%%%%%%%%%%%%%% 

\begin{document}

\maketitle

%--  PROBLEM 1  ---------------------------------------------------
\begin{problem}[SA 3.1.1aei]
Which of the following are subspaces? Justify your answers.
\begin{enumerate}
\item[a.] $\{(x_1, x_2) \in \R^2: x_1 + x_2 = 1\}$
%% \item[d.] $\{\vx \in \R^3 : x_1^2 + x_2^2+ x_3^2 = 1\}$
\item[e.] $\{\vx \in \R^3 : x_1^2 + x_2^2+ x_3^2 = 0\}$
\item[i.] $\{\vx \in \R^3: \vx = \begin{bmatrix*}[r] 2\\4\\-1\end{bmatrix*} 
+s\begin{bmatrix*}[r] 2\\1\\1\end{bmatrix*} 
+t\begin{bmatrix*}[r] 1\\2\\-1\end{bmatrix*} \text{ for some } s, t \in R\}$
\end{enumerate}
\end{problem}
%% \begin{solution}
%% (uncomment these 3 lines and type your solution here)
%% \end{solution}
\newpage

%--  PROBLEM 2  ---------------------------------------------------
\begin{problem}[SA 3.1.2cd]
Decide whether each of the following collections of vectors spans $\R^3$.
\begin{enumerate}
\item[c.] $\{(1, 0, 1), (1, -1, 1), (3, 5, 3), (2, 3, 2)\}$
\item[d.] $\{(1, 0, -1), (2, 1, 1), (0, 1, 5)\}$
\end{enumerate}
\end{problem}
%% \begin{solution}
%% (uncomment these 3 lines and type your solution here)
%% \end{solution}
\newpage


%--  PROBLEM 3  ---------------------------------------------------
\begin{problem}[SA 3.1.6]
    Let $U$ and $V$ be subspaces of $\R^n$. 
    The \emph{intersection} and \emph{union} of $U$ and $V$ are defined, respectively, as follows:
    \[U \cap V := \{\vx \in \R^n : \vx \in U \text{ and } \vx \in V \}\quad \text{ and } \quad
    U \cup V := \{\vx \in \R^n : \vx \in U \text{ or } \vx \in V\},\]
  \begin{enumerate}[a.]
  \item Show that $U \cap V$ is a subspace of $\R^n$. Give two examples.
  \item Is $U \cup V$ a subspace of $\R^n$? Give a proof or counterexample.
  \end{enumerate}
\end{problem}
%% \begin{solution}
%% (uncomment these 3 lines and type your solution here)
%% \end{solution}
\newpage

%--  PROBLEM 4  ---------------------------------------------------
\begin{problem}[SA 3.1.7]
Let $U$ and $V$ be subspaces of $\R^n$. We define the \emph{sum} of $U$ and $V$ 
to be
\[
U + V := \{\vx \in \R^n \mid \vx = \vu + \vv \text{ for some $\vu \in U$ and $\vv \in V$}\}.
\]
More simply, $U + V = \{\vu + \vv \mid \vu \in U \text{ and } \vv \in V\}$.\\
\\
Prove that if $U$ and $V$ are subspaces of $\R^n$ and $W$ is a subspace of 
$\R^n$ containing all the vectors of $U$ and all the vectors of $V$ 
then $U + V \subseteq W$. That is, prove
\[
U \leq W \; \text{ and } \; V \leq W \quad \text{ implies } \quad 
U + V \leq W.
\] 
This means that $U + V$ is the smallest subspace containing both $U$ and $V$.
\end{problem}
%% \begin{solution}
%% (uncomment these 3 lines and type your solution here)
%% \end{solution}
\newpage
%--  PROBLEM 5 ---------------------------------------------------
\begin{problem}[SA 3.1.9ad] 
Determine the intersection of the subspaces $\mathcal{P}_1$ and $\mathcal{P}_2$ in each case:
  \begin{enumerate}[a.]
  \item $\mathcal{P}_1 =  \Span \{(1, 0, 1), (2, 1, 2)\}$, $\mathcal{P}_2= \Span\{(1, -1, 0), (1, 3, 2)\}$.
  \item[d.] $\mathcal{P}_1 =  \Span \{(1, 1, 0, 1), (0, 1, 1, 0)\}$,
    $\mathcal{P}_2= \Span\{((0, 0, 1, 1), (1, 1, 0, 0)\}$.
  \end{enumerate}
\end{problem}

\newpage
\noindent The last required exercise of this assignment is Problem~\ref{prob:6},
which appears on the next page.
The (unnumbered) problems on this page are recommended but optional.
They merely test whether you know the definition of orthogonal complement 
(and thus prepare you for Problem~\ref{prob:6}).
We will cover orthogonal complements in lecture, but in case we don't get to it 
before you come to this part of the homework, here's the definition.
If $V$ is a subspace of $\R^n$, then the \emph{orthogonal complement} of $V$ in $\R^n$ 
is denoted by $V^\bot$ and is defined as follows:
\[V^\bot := \{ \vx \in \R^n \mid \vx \cdot \vv = 0 \text{ for all $\vv\in V$}\}.\]
\probskip

%--  PROBLEM  ---------------------------------------------------
\begin{prob}[SA 3.1.10]
Let $V \leq \R^n$ be a subspace. Show that $V \cap V^\bot = \{\vzero \}$.
\end{prob}
\probskip

%--  PROBLEM  ---------------------------------------------------

\begin{prob}[SA 3.1.11-2]
Suppose $V$ and $W$ are \emph{orthogonal subspaces} of $\R^n$, that is, 
$\vv \cdot \vw = 0$ for every $\vv \in V$ and every $\vw \in W$. 
\begin{enumerate}
\item Prove that $V \subseteq W^\bot$.
\item Prove that $V \cap W = \{\vzero\}$.
\end{enumerate}
\end{problem}
\newpage


%--  PROBLEM 6  ---------------------------------------------------
\begin{problem}[SA 3.1.14]
\label{prob:6}
Let $V$ and $W$ be subspaces of $\R^n$ with the property that 
$V \subseteq W$. Prove that $W^{\bot} \subseteq V^\bot$.
\end{problem}
%% \begin{solution}
%% (uncomment these 3 lines and type your solution here)
%% \end{solution}

\end{document}
\noindent {\bf Recommended Exercises} (students are not required to submit solutions
for the following)

\medskip


\probskip
\noindent
\begin{prob}[SA 5.1.9a]
  Suppose $A\in \R^{k\times k}$ and
  $B\in \R^{\ell \times \ell}$. Prove that
  \[
  \det
 \left[ 
    \begin{array}{c|c}
      A &B \\ 
      \hline
      O & D
    \end{array}
    \right] = \det A \det D.
 \]
\end{prob}

